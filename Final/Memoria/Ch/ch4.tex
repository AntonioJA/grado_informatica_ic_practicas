\chapterimage{chapter_head_1} % Chapter heading image

\chapter{Manual de uso}

Al ejecutar el sistema, se nos presenta el siguiente menú:

\begin{bashcode}
  ¿Qué te ocurre?

      1) Creo que tengo una ETS, porque tengo síntomas.
      2) Creo que podría tener una ETS, pero no presento síntomas.
      3) Me gustaría obtener información sobre las ETS.
      4) Salir.
  Insert (1 2 3):
\end{bashcode}

Si se selecciona 1, por ejemplo, que corresponde con el tipo de usuario 1, se mostrará lo siguiente:

\begin{bashcode}
  Selecciona algunos de los siguiente síntomas

    1) Dolor/escozor al orinar o al tener relaciones.
    2) Dolor de garganta.
    3) Algún tipo de erupción/berruga/úlcera.
    4) Terminar.
Insert (1 2 3 4):
\end{bashcode}

A partir de aquí, se define el tipo de usuario que interactuará con el sistema, y se procederá a sumunistrarle la información necesaria.
